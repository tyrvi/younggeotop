\documentclass[12pt]{article}
\synctex=1

% setup packages
\usepackage{amsmath,amssymb,amsthm,mathtools}
% \usepackage[hmargin={2cm,2cm},height=26cm]{geometry}
% \usepackage[a4paper,lmargin=2cm,rmargin=2cm]{geometry}
% \usepackage[a4paper,hmargin={2cm,2cm}]{geometry}
\usepackage[utf8]{inputenc}
\usepackage{cite}
\usepackage{imakeidx} % for index
\usepackage[hyperfootnotes=false]{hyperref}
\hypersetup{
  bookmarks=true,
  hyperindex=true,
  colorlinks=true,
  linkcolor=blue,
}
\usepackage{motivic}

% label things within a given section
\numberwithin{equation}{section}
% \numberwithin{align}{section}
\numberwithin{lemma}{section}
\numberwithin{theorem}{section}
\numberwithin{proposition}{section}
\numberwithin{corollary}{section}
\numberwithin{definition}{section}
\numberwithin{example}{section}
\numberwithin{remark}{section}

% start section numbering a 0
\setcounter{section}{-1}

\newcommand{\name}{}
\newcommand{\thedate}{}
\newcommand{\course}{Motivic Homotopy Theory}
\newcommand{\assignment}{}
\makeindex


\begin{document}
\title{\course\\ \assignment}
\author{\name{}}
\date{\thedate}
\maketitle

\tableofcontents

%-------------------------------------------------------------------------------

\section*{Introduction}
\addcontentsline{toc}{section}{Introduction}

Notes from the
\href{https://sites.google.com/view/maxime-ramzi-en/geotop-junior-seminar?authuser=0}{Spring
  2021 GeoTop junior seminar} at the University of Copenhagen. The
course primarily follows the
\href{https://math.mit.edu/events/talbot/index.php?year=2014}{Talbot
  seminar}.

\subsection*{Notation and Conventions}

Since we are primarily working in the world of algebraic geometry,
that is, with schemes we say that a topological space $X$
\emph{quasi-compact} if every open cover has a finite subcover and
reserve the term \emph{compact} for the case when $X$ is quasi-compact
and Hausdorff.

\section{Background from Algebraic Geometry}

\begin{definition}
  Let $f : X \to Y$ be a map of topological spaces, then $f$ is
  \emph{quasi-compact} if $f^{-1}(V)$ is quasi-compact for every
  quasi-compact open $V \subset Y$.
\end{definition}

\begin{lemma}{\emph{(\href{https://stacks.math.columbia.edu/tag/01K2}{Tag 01K2})}}
  Let $f : X \to S$ be a morphism of schemes, then the following are
  equivalent:
  \begin{enumerate}[label=(\arabic*)]
  \item $f : X \to S$ is quasi-compact,
  \item the inverse image of every affine open is quasi-compact,
  \item there exists an affine open cover $S = \cup_{i \in I} U_i$
    such that $f^{-1}(U_i)$ is quasi-compact for all $i$.
  \end{enumerate}
\end{lemma}

In topology one has the theorem that a space $X$ is Hausdorff if and
only if the diagonal
\begin{align*}
  \Delta : X &\to X \times X \\
  x &\mapsto (x, x)
\end{align*}
is closed. In algebraic geometry given $f : X \to S$ we have a
diagonal map obtained by the fiber product
$\Delta_{X/S} : X \to X \times_{S} X$. That is, $\Delta_{X/S}$ is the
unique morphism of schemes such that
$\pr[1] \circ \Delta_{X/S} = \id[X]$ and
$\pr[2] \circ \Delta_{X/S} = \id[X]$. One can show that $\Delta_{X/S}$
is an immersion (\href{https://stacks.math.columbia.edu/tag/01KH}{Tag
  01KH}).

\begin{definition}
  Let $f : X \to S$ be a morphism of schemes.
  \begin{enumerate}[label=(\arabic*)]
  \item $f$ is said to be separated if the diagonal $\Delta_{X/S}$
    along $f$ is a closed immersion.
  \item $f$ is said to be quasi-separated if the diagonal
    $\Delta_{X/S}$ is a quasi-compact morphism.
  \item A scheme $Y$ is separated if the morphism $Y \to \Spec \Z$ is separated.
  \item A scheme $Y$ is quasi-separated if the morphism
    $Y \to \Spec \Z$ is quasi-separated.
  \end{enumerate}
\end{definition}

Recall that a map of rings $R \to A$ is said to be of finite type if
$A$ is isomorphic to a quotient of $R[x_1, \dots, x_n]$ as an
$R$-algebra, that is, if $A$ is a finitely generated $R$-algebra.

\begin{definition}{(\href{https://stacks.math.columbia.edu/tag/01T0}{Tag 01T0})}
  Let $f : X \to S$ be a morphism of schemes, then
  \begin{enumerate}[label=(\arabic*)]
  \item $f$ is said to be of \emph{finite type at $x \in X$} if there
    exists an affine open neighborhood $\Spec A = U \subset X$ of $x$
    and an affine open $\Spec R = V \subset S$ with $f(U) \subset V$ such that the induced ring map $R \to A$ is of finite type.
  \item $f$ is \emph{locally of finite type} if $f$ is of finite type
    for all $x \in X$.
  \item $f$ is of \emph{finite type} it is locally of finite type and
    quasi-compact.
  \end{enumerate}
\end{definition}

\begin{definition}
  Let $X$ be a scheme, then $X$ is \emph{Noetherian} if it admits a
  finite cover by affine Schemes, $\Spec A_i$, where each $A_i$ is a
  Noetherian ring.
\end{definition}

\begin{lemma}{(\emph{\href{https://stacks.math.columbia.edu/tag/01T6}{Lemma
        01T6}})}
  Let $f : X \to S$ be a morphism. If $S$ is (locally) Noetherian and
  $f$ (locally) of finite type then $X$ is (locally) Noetherian.
\end{lemma}

Let $f : X \to S$ be a morphism of schemes, then we refer to
\href{https://stacks.math.columbia.edu/tag/01V4}{Tag 01V4} for a
general definition of smoothness. However, since we will be working
with $f : X \to S$ of finite type with $S$ a Noetherian scheme we can
instead refer to \href{https://stacks.math.columbia.edu/tag/02HW}{Tag
  02HW} for definition of Smoothness. For definitions of etale
morphisms see \href{https://stacks.math.columbia.edu/tag/02GH}{Tag
  02GH} and \href{https://stacks.math.columbia.edu/tag/0257}{Tag 0257}
or \href{https://stacks.math.columbia.edu/tag/024J}{Chapter 024J}

\begin{definition}
  Let $X$ be a scheme and $x$ a point in $X$. Let $U = \Spec A$ be an
  affine neighborhood of $x$. Let $P_x$ be the prime ideal
  corresponding to $x$, then the localization $A_{P_x}$ is a local
  ring with maximal ideal $P_x \cdot A_{P_x}$. The \emph{residue
    field} of $x$ is the field
  \begin{equation*}
    k(x) := A_{P}/P \cdot A_{P}.
  \end{equation*}
  Note that there is an isomorphism $k(x) \cong \text{Frac}(A/P)$
  % TODO: see Bosch exercise 1.2-5
  We call the point $x$ $K$-rational for a certain field $K$ if
  $K \cong k(x)$.
\end{definition}

Let $A$ be a local ring with maximal ideal $\mmm$ and residue field
$k$. Let $a \mapsto \ol{a}$ denote the map $A \to k$ and
$f \mapsto \ol{f}$ denote the map $A[T] \to k[T]$. If $A$ is a
complete\footnote{Any discrete valuation ring may be given a metric
  $\abs{x - y} = 2^{-\nu(x - y)}$ and we may take the completion with
  respect to this metric. Alternatively, we can consider the
  \href{https://en.wikipedia.org/wiki/Completion_of_a_ring}{completion}
  of the ring considered as a topological ring since local rings may
  be given a
  \href{https://en.wikipedia.org/wiki/Local_ring\#Some_facts_and_definitions}{natural
    topology}}
\href{https://en.wikipedia.org/wiki/Discrete_valuation_ring}{discrete
  valuation ring}, then Hensel's lemma states: If f is a monic
polynomial with coefficients in $A$ such that $\ol{f}$ factors as
$\ol{f} = g_0h_0$ with $g_0$ and $h_0$ monic and
coprime\footnote{i.e. the ideal $(g_0, h_0) = (1)$ generates the whole
  ring}, then $f$ factors as $f = gh$ with $g$ and $h$ monic such that
$\ol{g} = g_0$ and $\ol{h} = h_0$.

\begin{definition}
  Let $A$ be a local ring with maximal ideal $\mmm$, then $A$ is said
  to be Henselian if Hensel's lemma holds for $A$.
\end{definition}

Given a ring a local ring $(A, \mmm, k)$ there is a universal ring
$A^{h}$ called the Henselization of $A$
(\href{https://stacks.math.columbia.edu/tag/0BSK}{Tag 0BSK},
\href{https://stacks.math.columbia.edu/tag/07QL}{Tag 07QL},
\href{https://stacks.math.columbia.edu/tag/03QD}{Tag 03QD}). That is,
the assignment $A \mapsto A^{h}$ is functorial.

% TODO: etale morphisms

% TODO: reduced structures on a scheme

% TODO: generic points of a scheme

%-------------------------------------------------------------------------------

\section{The Nisnevich Topology and Universal Properties of the Stable
  and Unstable Motivic Homotopy Categories}

In this talk we introduce the Nisnevich topology for a scheme and
discuss some of it's properties (including, hopefully, descent) as it
is central to the definition of the Motivic homotopy category. Then we
introduce the (un)stable Motivic homotopy categories and discuss/prove
their universal properties following~\cite{robalo2012noncommutative}
sections $5.1 - 5.3$. Finally, note that in many of the constructions
there are the usual subtle set-theoretic issues that must be dealt
with in order to avoid Russell type paradox. We will generally ignore
any such subtleties in the interest of exposition as all the issues
can be dealt with through the use of universes and enlarging the
universe when necessary.

\subsection{Preliminaries}

We begin by recalling the notion of Grothendieck (pre)topologies as
well as

\subsubsection*{Grothendieck (Pre)Topologies}

We begin by discussing Grothendieck (pre)topologies. Although we
attempt to work as much as possible in the setting of
$\infty$-categories the $\infty$-categorical generalization of a
Grothendieck topology is in some sense no generalization at
all\footnote{This is due to the fact that there is a bijection between
  Grothendieck topologies on an $\infty$-category $\CC$ and
  Grothendieck topologies on $h\CC$. Additionally, for a $1$-category
  one has a canonical equivalence $N(\CC_{/C}) \simeq
  N(\CC)_{/C}$. Since we will primarily be interested in the nerve of
  a $1$-category this has no effect. See HTT Remark 6.2.2.3 for
  details.} so for ease of exposition we discuss Grothendieck
topologies purely in the setting of $1$-categories.

\begin{definition}
  Let $\CC$ be a $1$-category. A \emph{Grothendieck pretopology $\tau$
    on $\CC$} consists of for each $C \in \CC$ a set $\Cov_{\tau}(C)$
  of \emph{covering families} where a covering is a set of morphisms
  $\{f_i : U_i \to C\}_{i \in I}$ in $\CC$ such that
  \begin{enumerate}[label=(\arabic*)]
  \item $\{\id[C]\}$ is in $\Cov_{\tau}(C)$ for all $C \in \CC$;
  \item for $\{f_i : U_i \to C\} \in \Cov_{\tau}(C)$ and $g : D \to C$
    a morphism in $\CC$ the fiber product $U_i \times_C D$ exists and
    the natural projection $\{\pr[2] : U_i \times_C D \to D\}$ is in
    $\Cov_{\tau}(D)$;
  \item if $\{f_j : U_j \to C\}_{j \in J}$ is in $\Cov_{\tau}(C)$ and
    $\{g_{ij} : V_{ij} \to U_j\} \in \Cov_{\tau}(U_j)$ for all $j$,
    then $\{f_j \circ g_{ij} : V_{ij} \to C\} \in \Cov_{\tau}(C)$.
  \end{enumerate}
\end{definition}

% TODO: discuss Grothendieck topologies
\begin{definition}
  \begin{enumerate}[label=(\arabic*)]
  \item Let $\CC$ be a $1$-category and $C \in \CC$. A \emph{sieve on
      $C$} is a subfunctor of $\y_C := \Hom_{\CC}(-, C)$, that is, a
    set of objects $\SS$ of $\CC_{/C}$ such that for each
    $f : D \to C$ in $\SS$ and each morphism $g_i : D_i \to D$ in
    $\CC$ the composition $f \circ g_i : D_i \to C$ is in $\SS$.
  \item Let $\FF := \{f_i : D_i \to C\}$ be a collection of morphisms
    in $\CC$. The \emph{sieve generated by $\FF$} is the collection of
    morphisms $h : D \to C$ which factor as $h = f_i \circ g$ for some
    $f_i \in \FF$.
  \item If $\SS$ is a sieve over $C$ and $f_i : D_i \to C$ is a
    morphism in $\CC$, then the restriction $\SS_{D_i}$ of $\SS$ to
    $D_i$ is the sieve over $D_i$ generated by the maps
    $g : D \to D_i$ such that $f_i \circ g \in \SS$.
  \end{enumerate}
\end{definition}

\begin{definition}
  Let $\CC$ be a $1$-category. A \emph{Grothendieck topology on $\CC$}
  consists of for each $C \in \CC$ a family $\SS(C)$ of sieves over
  $C$ which we call \emph{covering sieves} such that:
  \begin{enumerate}[label=(\arabic*)]
  \item The sieve $\{\id[C]\}$ is a covering sieve of $C$.
  \item If $\SS$ is a covering sieve of $C$ and $f : D \to C$ is a
    morphism in $\CC$, then $\SS_{D}$ is a covering sieve of $D$.
  \item Let $\SS$ be a covering sieve of $C$. If $\TT$ is a covering
    sieve of $C$ such that for each $f : D \to C$ in $\SS$ the
    restriction $\TT_D$ is a covering sieve of $D$, then $\TT$ is a
    covering sieve of $C$.
  \end{enumerate}
  We refer to a category $\CC$ with a Grothendieck topology $\tau$ as
  a site denoted $\CC_\tau$ and where $\abs{\CC_{\tau}} = \CC$ denotes
  the underlying category.
\end{definition}

Observe that if for each $C \in \CC$ we have a family $\SS_0(C)$ of
sieves over $C$, then there is a minimal Grothendieck topology
$C \mapsto \SS(C)$ with $S_0(C) \subset \SS(C)$ for all $C$. We call
this the \emph{Grothendieck topology generated by $\SS_0(C)$}.

Given a Grothendieck pretopology $\tau$ on $\CC$, then we may obtain a
Grothendieck topology on $\CC$ in the following way. Since $\tau$ is a
Grothendieck pretopology, then for each $C \in \CC$ there is a
covering family $\{f_i : U_i \to C\} \in \Cov_{\tau}(C)$. Thus, we may
consider the sieve, $\SS_I$, generated by the covering family
$\{f_i : U_i \to C\}$. Then taking the collection of all sieves
generates a Grothendieck topology on $\CC$.

\begin{remark}
  Let $\tau$ be a Grothendieck topology on a category $\CC$. Given a
  full subcategory $\CC_0$ of $\CC$, then it will be important to
  consider a Grothendieck topology on $\CC_0$ induced by $\tau$. This
  is done in the following way.

  Let $\CC_0$ be a full subcategory of $\CC$ such that if
  $X \in \CC_0$, $g : Y \to X$ is a morphism in $\CC_0$, and
  $\{f_i : U_i \to X\}_{i \in I}$ is in $\Cov_{\tau}(X)$, then each
  $U_i$ is in $\CC_0$ and each fiber product $Y \times_X U_i$ is in
  $\CC_0$. Then we obtain a Grothendieck topology $\tau_{\CC_0}$ on
  $\CC_0$ by restricting $\tau$ to $\CC_0$, that is, setting
  \begin{equation*}
    \Cov_{\tau_0}(X) := \Cov_{\tau}(X)
  \end{equation*}
  for $X \in \CC_0$.

  The case we will primarily be working with where this is important
  is where we have $\Sm^{ft}(S) := \Sm^{ft}_{/S}$ the full subcategory
  of $\Sch_{/S}$ consisting of smooth $S$-schemes of finite type. The
  Zariski, etale, and Nisnevich topologies will, then restrict to
  Grothendieck topologies on $\Sm^{ft}(S)$ giving sites
  $\Sm^{ft}(S)_{Zar}$, $\Sm^{ft}(S)_{et}$, and $\Sm^{ft}(S)_{Nis}$.
\end{remark}

\begin{example}
  The \emph{indiscrete} topology on a category $\CC$ is the topology
  $ind$ with $\Cov_{ind}(X) = \{\{\id[X]\}\}$.
\end{example}

\begin{example}\label{ex:topological-site}
  The canonical example is the following: let $X$ be a topological
  space and $\CC = \UU(X)$ be the poset of open subsets of $X$. Then
  for $U \subset X$ open, $\{f_i : U_i \to U\}$ is in $\Cov(U)$ if
  $U = \cup_{i} U_i$.

  A similar example is let $\Top$ be the category of topological
  spaces. Let $\Cov_{\Top}(U) := \Cov_{U}(U)$ gives a Grothendieck
  topology on $\Top$. Note that for $U_i \subset U$ open and
  $f : V \to U$ continuous the fiber product $V \times_{U} U_i$ is
  just the open subset $f^{-1}(U_i) \subset V$. In particular, this
  implies for $\{f_i : U_i \to U\} \in \Cov(U)$, the fiber product
  $U_i \times_U U_j$ is just the intersection $U_i \cap U_j$.
\end{example}

\begin{example}
  Let $X$ be a scheme and $\abs{X}$ denote the underlying topological
  space. Taking $\UU(\abs{X})$ as in example~\ref{ex:topological-site}
  with the same covering families, that is, $\{f_i : U_i \to U\}$ is
  in $\Cov(U)$ if $U = \cup_{i}U_i$, then this gives the Zariski
  topology on $X$. We let $X_{Zar}$ denote the site with underlying
  category $\UU(X)$. Again as in example~\ref{ex:topological-site} we
  may define a Grothendieck topology on $\Sch$, the category of
  schemes over a scheme $X$, denoted $\Sch_{Zar}$
\end{example}

\begin{example}
  Let $X$ be a scheme. The etale site $X_{et}$ is the site with
  underlying category $\Sch^{ft,et}_{/X}$ consisting of $f : U \to X$
  such that $f$ is an finite type etale morphism. For a given $U$ a
  covering family $\{f_i : U_i \to U\}$ is a set of finite type etale
  morphisms such that $\coprod_{i} \abs{U_i} \to \abs{U}$ is
  surjective. Again we may define a site $\Sch_{et}$ on the category
  $\Sch$ of schemes.
\end{example}
% TODO: additional potential examples: h-topology, qfh-topology,
% cdh-topology, indiscrete topology

\subsubsection*{Presheaves and Sheaves on a Site}

Now given a site $\CC_\tau$ we may consider $\AA$-valued presheaves on
$\CC_\tau$ by defining $\PSh(\CC_\tau) := \Fun(\CC^{op}, \AA)$. To
define $\AA$-valued sheaves requires us to be able to state the sheaf
axiom which requires $\AA$ to admit products. Recall an equalizer is the limit of a diagram
\begin{equation*}
  \begin{tikzcd}
    \bullet \arrow[r, shift right=2] \arrow[r, shift left=2] & \bullet
  \end{tikzcd}
\end{equation*}
Concretely, the equalizer of two parallel morphisms $f, g : A \to B$
is an object $E$ and a morphism $e : E \to A$ which is universal such
that $f \circ e = g \circ e$. That is, given any $z : Z \to A$ such
that $f \circ z = g \circ z$, then there exists a unique morphism
$u : Z \to E$ with $e \circ u = z$ as in the diagram
\begin{equation*}
  \begin{tikzcd}
    E \arrow[r, "e"]                          & A \arrow[r, "g"', shift right=2] \arrow[r, "f", shift left=2] & B \\
    Z \arrow[u, "u", dashed] \arrow[ru, "z"'] &                                                               &
  \end{tikzcd}
\end{equation*}
If $e : E \to A$ is the equalizer of $f$ and $g$ we will say that the
sequence
\begin{equation*}
  \begin{tikzcd}
    E \arrow[r, "e"] & A \arrow[r, "g"', shift right=2] \arrow[r, "f", shift left=2] & B
  \end{tikzcd}
\end{equation*}
is exact. Note that if $\AA$ is an abelian category this is equivalent
to requiring that the sequence
\begin{equation*}
  \begin{tikzcd}
    0 \arrow[r] & E \arrow[r, "e"] & A \arrow[r, "f - g"] & B
  \end{tikzcd}
\end{equation*}
is exact in the usual sense.

Hence, let $\AA$ be a category with products and let $F$ be an
$\AA$-valued presheaf on $\CC_\tau$ and
$\{f_i : U_i \to X\}_{i \in I} \in \Cov_{\tau}(X)$ for some
$X \in \CC$. Then we have ``restriction'' morphisms
\begin{align*}
  f_i^* : F(X) &\to F(U_i) \\
  \pr[1, i, j]^* : F(U_i) &\to F(U_i \times_X U_j) \\
  \pr[2, i, j]^* : F(U_j) &\to F(U_i \times_X U_j)
\end{align*}
and taking products we obtain the diagram in $\AA$
\begin{equation}\label{eq:sheaf-condition}
  \begin{tikzcd}
    F(X) \arrow[r, "\prod_i f_i^*"] & \prod_iF(U_i) \arrow[rr, "{\prod_j \pr[2, i, j]^*}"', shift right=2] \arrow[rr, "{\prod_i \pr[1, i, j]^*}", shift left=2] &  & {\prod_{i, j} F(U_i \times_X U_j)}
  \end{tikzcd}
\end{equation}

\begin{definition}
  Let $\AA$ be a category which has arbitrary (small) products and let
  $\tau$ be a Grothendieck pretopology on a small category $\CC$. An
  $\AA$-valued presheaf $F$ on $\CC$ is a sheaf for $\tau$ if for each
  covering $\{f_i : U_i \to X\} \in \Cov_\tau$ the sequence
  in~\ref{eq:sheaf-condition} is exact. Let $\Shv_\tau^{\AA}(\CC)$
  denote the category of $\AA$-valued sheaves on $\AA$ with respect to
  $\tau$ which is the full subcategory of $\PSh^{\AA}(\CC)$ with all
  objects sheaves. If the context is clear we write $\Shv(\CC)$.
\end{definition}

% TODO: simplicial sheaves?

\subsubsection*{Simplicial Presheaves and Sheaves}

In the case of motivic homotopy theory we will want to use presheaves
and sheaves of simplicial sets (spaces, anima, Kan complexes). Let
$\Delta$ denote the skeleton of the category of finite ordered sets
and order preserving maps, that is, $\Delta$ has object
$[n] = \{ 0 < 1 < \cdots < n\}$ and morphisms the order preserving
maps. Then define the category of simplicial sets as presheaves of
sets on $\Delta$, that is,
\begin{equation*}
  \sSet := \Fun(\Delta^{op}, \Set).
\end{equation*}
For a category $\CC$ we may make the identifications
\begin{equation*}
  \PSh(\CC, \sSet) = \PSh(\CC \times \Delta, \Set) = \PSh(\Delta, \PSh(\CC, \Set)).
\end{equation*}
so we define the category of simplicial presheaves as
\begin{equation*}
  \PSh(\CC, \sSet) = \Fun(\CC^{op}, \sSet) = \Fun(\CC^{op}, \Fun(\Delta^{op}, \Set)) = \PSh(\Delta, \Fun(\CC^{op}, \Set))
\end{equation*}

Similarly, if $\tau$ is a Grothendieck topology on $\CC$ we have an
induced topology on $\CC \times \Delta$ given by
$\Cov_{\tau}(X \times [n]) := i_n(\Cov_\tau(X))$ for each $X \in \CC$
where $i_n : \CC \to \CC \times \Delta$ is the inclusion functor
$i_n(X) := X \times [n]$, $i_n(f) = f \times \id[[n]]$. Thus, this
defines the category of sheaves on $\CC \times \Delta$ which may be
identified with the category of functors
$\Delta^{op} \to \Shv_\tau(\CC, \Set)$, that is, simplicial
sheaves. In other words a simplicial presheaf $n \mapsto F_n$ is a
simplicial sheaf if and only if each $F_n$ is a sheaf. It follows that
all the elementary results on presheaves and sheaves of sets extends
to presheaves and sheaves of simplicial sets through these
identifications.

\subsection{The Nisnevich Topology and Descent}

We now define the Nisnevich topology on a scheme $X$ and discuss
various properties.

\begin{definition}
  Let $U \to X$ be a morphism of schemes and let $x$ be a point in
  $X$. Then $f : U \to X$ is said to be \emph{completely decomposed at
    $x$} if there exists $u \in U$ with $f(u) = x$ such that the
  induced map of residue fields $k(x) \to k(u)$ is an isomorphism.
\end{definition}

One makes the observation that points $u$ over which $x$ is completely
decomposed are in bijection with lifts in the diagram
\begin{equation*}\label{eq:lift-condition}
  \begin{tikzcd}
    & U \arrow[d] \\
    \Spec k(x) \arrow[r] \arrow[ru, dashed] & X
  \end{tikzcd}
\end{equation*}
where the lift $\Spec k(x) \to U$ corresponding to $u$ is induced by
$\OO_{U, u} \to k(u) \cong k(x)$\footnote{Using $k(u) \cong k(x)$ to
  get a lift is clear by considering the usual diagram relating the
  spectrum of the residue field of a point and the scheme however the
  other direction is not immediately clear to me.}.
% TODO: double check this, but it should basically follow from writing
% down the right diagrams and thinking about it.
This fact can then allows the immediate deduction of the following:
\begin{lemma}
  Given a pullback square
  \begin{equation*}
    \begin{tikzcd}
      V \arrow[d, "\tilde{f}"'] \arrow[r] & U \arrow[d, "f"] \\
      Y \arrow[r, "g"]      & X
    \end{tikzcd}
  \end{equation*}
  such that $f(y) = x$, then if $f : U \to X$ is completely decomposed
  at $x$, then $\tilde{f} : V \to Y$ is completely decomposed at
  $y$. That is, being completely decomposed is stable under pullback.
\end{lemma}

\begin{definition}
  A collection, $\{f_i : U_i \to X\}_{i \in I}$, of morphisms of
  schemes is said to be a \emph{Nisnevich covering} if
  \begin{enumerate}[label=(\arabic*)]
  \item $I$ is finite
  \item each morphism $f_i$ is etale and of finite type
  \item For each $x \in X$ there exists $i \in I$ such that $f_i$ is
    completely decomposed at $x$.
  \end{enumerate}
\end{definition}

\begin{proposition}
  $\{f_i : U_i \to X\}_{i \in I}$ is a Nisnevich covering if and only if
  the single induced map $\coprod_{i \in I} U_i \to X$ is a Nisnevich
  covering.
  % TODO: double check this
\end{proposition}

\begin{proposition}
  Let $S$ be a scheme and $\CC$ a full subcategory of $\Sch_{/S}$
  such that the pullback in $\Sch_{/S}$ of a diagram
  \begin{equation*}
    \begin{tikzcd}
      & U \arrow[d, "p"] \\
      Y \arrow[r] & X
    \end{tikzcd}
  \end{equation*}
  in $\CC$ where $p$ is etale of finite type is in $\CC$. Then the
  class of Nisnevich coverings form a basis for a Grothendieck
  topology on $\CC$. We call the induced topology the \emph{Nisnevich}
  topology.
\end{proposition}

From the definition once can see that any Zariski cover is a Nisnevich
cover and any Nisnevich cover will be an etale cover. Hence, it
follows that the Zariski topology is coarser than the Nisenvich
topology is coarser then the etale topology. Recall that for a scheme
$X$, then the representable presheaf $X := \Hom(-, X)$ is a sheaf in
the etale topology. Hence, it follows that every representable
presheaf in the Nisnevich topology is a sheaf, that is, for an
appropriate site $\CC_\tau$ with $\tau = Zar, Nis, et$ we have the
sequence of fully faithful embeddings
\begin{equation*}
  \CC \subset \Shv(\CC_{et}) \subset \Shv(\CC_{Nis}) \subset \Shv(\CC_{Zar}) \subset \PSh(\CC).
\end{equation*}

One additionally has the following proposition which provides an
alternative characterization of the Nisnevich topology.
\begin{proposition}\label{prop:alternative-nisnevich-condition}
  Let $U \to X$ be an etale morphism and $x \in X$. Then the following
  are equivalent:
  \begin{enumerate}[label=(\arabic*)]
  \item $U \to X$ is competely decomposed at $x$;
  \item The morphism
    $U \times_X \Spec \OO^h_{X, x} \to \Spec \OO^h_{X, x}$ has a
    section.
  \end{enumerate}
\end{proposition}

In some sense this says that the ``points'' in the Nisnevich topology
are determined by Hensel local schemes. To prove the proposition one
uses the following theorem from~\cite{milne1980etale}.

\begin{theorem}{\emph{(\cite{milne1980etale}) Theorem I.4.2(a), (b), (d)}}
  Let $A$ be a local ring with maximal ideal $\mmm$ and residue field
  $k$. Let $x$ be the closed point of $X = \Spec A$, then the
  following are equivalent:
  \begin{enumerate}[label=(\alph*)]
  \item $A$ is Henselian
  \item any finite $A$-algebra $B$ is a direct product of local rings
    $B = \prod_{i \in I} B_i$ where the $B_i$ are necessarily
    isomorphic to the rings $B_{\mmm_i}$ with the $\mmm_i$ the maximal
    ideals of $B$.
  \item If $f : Y \to X$ is etale and there is a point $y \in Y$ such
    that $f(y) = x$ and $k(y) \cong k(x)$, then $f$ has a section
    $s : X \to Y$.
  \end{enumerate}
\end{theorem}

\begin{proof}{(of Proposition~\ref{prop:alternative-nisnevich-condition})}
  Assume (1) holds, then the morphism
  $U \times_X \Spec \OO^h_{X, x} \to \Spec \OO^h_{X, x}$ is etale by
  properties of Henselization and pullbacks and completely decomposed
  at the closed points of $\Spec \OO^h_{X, x}$ (Why?). Thus, it
  follows by~\cite{milne1980etale} Theorem I.4.2(d) that there is a
  section so (2) holds.

  For $(2) \implies (1)$ consider the extension of scalars using
  $\OO^{h}_{X, x} \to k(x)$, then one see that
  $U \times_X \Spec k(x) \to \Spec k(x)$ has a section which is
  equivalent to a lift in~\eqref{eq:lift-condition}.
  % TODO: double check this
\end{proof}

\begin{definition}\label{def:elementary-distinguished-square}
  A commutative square
  \begin{equation*}\label{eq:elementary-distinguished-square}
    \begin{tikzcd}
      U \times_X V \arrow[d] \arrow[r] & V \arrow[d, "p"] \\
      U \arrow[r, "i"] & X
    \end{tikzcd}
  \end{equation*}
  is said to be an \emph{elementary distinguished square} if
  \begin{enumerate}[label=(\arabic*)]
  \item the square is a pullback;
  \item $i : U \xhra{i} X$ is an open immersion;
  \item $p : V \to X$ is etale;
  \item the canonical projection $p^{-1}(Z) \to Z$ is an isomorphism
    where $Z := X \setminus U$ and $p^{-1}(Z)$ have the
    \href{https://stacks.math.columbia.edu/tag/01J4}{reduced
      structure} of closed subschemes.
  \end{enumerate}
  Moreover, one can see that an elementary distinguished square is in
  fact a pushout.

  An important fact is that $\{i : U \xhra{} X, p : V \to X\}$ is a
  Nisenvich covering and we will see later (Theorem~\ref{thm:descent})
  that they in fact for a basis for the Nisnevich topology.
\end{definition}

\begin{lemma}
  Suppose $p : V \to X$ is etale, $Z \subseteq X$ is a closed subset,
  and $Z$ and $p^{-1}(Z)$ are given the reduced structures. Then the
  square
  \begin{equation*}
    \begin{tikzcd}
      p^{-1}(Z) \arrow[d] \arrow[r] & V \arrow[d, "p"] \\
      Z \arrow[r, hook]                             & X
    \end{tikzcd}
  \end{equation*}
  is a pullback.
\end{lemma}

\begin{proof}
  Indeed the pullback $V \times_X Z \to V$ is a closed immersion with
  underlying closed subset $p^{-1}(Z)$. Additionally, $V \times_X Z$
  is reduced since an etale scheme over a reduced scheme is reduced
  (ref?). Hence, the uniqueness of reduced subschemes on a given
  closed subset (ref?) ensures that the map
  $p^{-1}(Z) \to V \times_{X} Z$ induced by the usual pullback square
  is an isomorphism.
\end{proof}

% Observe that a distinguished square determines a Nisnevich covering by
% taking $V \to X$ and $U \to X$.
% TODO: why?

\begin{proposition}
  % proposition 1.3
  Let $S$ be a scheme and $\CC$ a full subcategory of $\Sch_{/S}$,
  then the elementary distinguished squares
  (\ref{def:elementary-distinguished-square}) are pushouts in
  $\Shv(\CC_{Nis})$.
\end{proposition}

\begin{proof}
  % TODO
\end{proof}

\begin{corollary}
  % corollary 1.4
  Let $\CC$ be a full subcategory of $\Sch_{/S}$. Given an elementary
  distinguished square in $\CC$ and if $F$ is a sheaf of abelian
  groups on $\CC_{Nis}$, then there is a natural associated long exact
  Mayer-Vietoris sequence
\end{corollary}

\begin{proof}
  % TODO
\end{proof}

\begin{lemma}
  % lemma 1.5
  Let $p : U \to X$ be etale and $U$, $X$ Noetherian schemes. If $p$
  is completely decomposed at every generic point of $X$, then
  $U \to X$ has a rational section. That is, there exists a dense open
  subset $X' \subset X$ such that $p : p^{-1}(X') \to X'$ has a section.
  % TODO: what is a rational section
\end{lemma}

\begin{proof}
  % TODO
\end{proof}

\begin{proposition}
  % proposition 1.6
  Let $p : U \to X$ be a Nisnevich covering with $U$ and $X$
  Noetherian. Then there is a finite filtration
  \begin{equation*}
    \emptyset = Z_n \subseteq Z_{n-1} \supseteq \cdots \subseteq Z_0 = X
  \end{equation*}
  of $X$ by closed subsets such that for each $0 \leq i \leq n$ the
  map $^{-1}(Z_i \setminus Z_{i+1}) \to Z_i \setminus Z_{i+1}$ has a
  section where for $i \geq 1$ $Z_i$ and $p^{-1}(Z_i)$ are given the
  reduced structures.
\end{proposition}

\begin{theorem}\label{thm:descent}
  % theorem 1.7
  Let $S$ be a scheme and $\CC$ a full subcategory of $\Sch_{/S}$
  consisting of Noetherian schemes. Let $\AA$ be any category (should
  maybe be a presheaf of sets?), then a presheaf
  $F \in \PSh(\CC, \AA)$ is a sheaf in the Nisnevich topology on $\CC$
  if and only if
  \begin{enumerate}[label=(\arabic*)]
  \item $F(\emptyset)$ is terminal in $\AA$;
  \item for any $X \in \CC$ and any elementary distinguished square
    (\eqref{eq:elementary-distinguished-square}) the induced square
    \begin{equation*}
      \begin{tikzcd}
        F(X) \arrow[d] \arrow[r] & F(U) \arrow[d, "F(i)"] \\
        F(V) \arrow[r, "F(p)"] & F(U \times_X V)
      \end{tikzcd}
    \end{equation*}
    is a pullback.
  \end{enumerate}
\end{theorem}

\begin{proof}
  % TODO
\end{proof}

\subsection{The Motivic Homotopy Category}

We now have the necessary knowledge about the Nisnevich topology to
construct the motivic homotopy category. The main goal of the
construction of the motivic homotopy category and its stable variant
is to allow the general methods of abstract homotopy theory to be
applied to the algebro-geometric world of schemes. The issue of course
with simply applying the usual tools of homotopy theory and algebraic
topology to the underlying topological space of a scheme is that this
ignores a great deal of the structure which is part of a
scheme. Additionally, the ``natural'' topology, the Zariski topology,
on a scheme is incredibly coarse and ignores the ``true'' topological
nature of many of the objects.

Fix $S$ a Noetherian scheme and let $\Sm^{ft}(S)$ denote the category
of smooth separated schemes of finite type over $S$. The general idea
to be able to do homotopy theory with schemes is that we should think
of the affine line $\A^1$ as an interval. However, many of the usual
constructions of homotopy theory rely on the existence of (co)limits
which is not true in the case of schemes. In order to deal with this
one passes to the formal completion of the appropriate category
(i.e. presheaves of spaces). This is however still not quite right as
we would like $\A^1$ to be contractible as well any scheme $X$ which
is a union of two open subschemes $U$ and $V$ should still union to
give $X$ in the new category.

We denote by $\HH(S)$ what will be the $\infty$-category underlying
the $\A^1$-model category constructed by Morel and Voevodsky.

The construction proceeds as follows. Fix a Noetherian scheme and let
$\Sm^{ft}(S)$ denote the category of smooth separated schemes of
finite type over $S$. Consider $\Sm^{ft}(S)$ as an $\infty$-category
by $\CC := N(\Sm^{ft}(S))$, then with the Nisnevich topology it
becomes an $\infty$-site (HTT Definition 6.2.2.1). As we have seen the
families of morphisms $\{V \xra{i} X, U \xra{p} X\}$ form a basis for the
Nisnevich topology (Theorem~\ref{thm:descent}).

Let $\SS$ denote the $\infty$-category of spaces (homotopy types, Kan
complexes, anima etc.). For a simplicial set $S$ we let
$\PSh(S) := \Fun(S^{op}, \SS)$ denote the $\infty$-category of
presheaves of spaces on the simplicial set $S$. To obtain all
(co)limits we pass to $\PSh(\CC)$ which has all (small) (co)limits
(HTT Corollary 5.1.2.4)\footnote{HTT Proposition 4.2.4.4 ensures that
  this category may be identified with the underlying
  $\infty$-category of the model category of simplicial presheaves on
  $\Sm^{ft}(S)$ which is the category used in~\cite{morel1999}. So the
  homotopy theory we end up with will be the same.}.

Recall the $\infty$-categorical Yoneda embedding (HTT 5.1.3) ensures
that we have a fully faithful map of $\infty$-categories
$j : \CC \to \PSh(\CC)$ and we identify $X \in \CC$ with $j(X)$ as
usual. By Theorem~\ref{thm:descent} we get that $F \in \PSh(\CC)$ is a
sheaf if and only if it maps Nisnevich squares to pullback squares. It
follows that since elementary distinguished squares are
pushouts\footnote{This fact should follow from gluing pullback
  squares.} (Definition~\ref{def:elementary-distinguished-square}),
then every representable presheaf $j(X)$ is a sheaf. Write
$\Shv_{Nis}(\CC) \subseteq \PSh(\CC)$ for the $\infty$-category of
sheaves in the Nisnevich topology on $\CC$. This has a left adjoint
$a_{Nis}$ sheafification which is an exact functor\footnote{In
  particular, is a topological localization at the collection of all
  monomorphisms $i : U \to j(C)$ corresponding to covering sieves on
  $C \in \CC$ which implies $\Shv_{Nis}(\CC)$ is an $\infty$-topos
  (HTT Definition 6.1.0.4).}  (HTT Lemma 6.2.2.7) and
$\Shv_{Nis}(\CC)$ is an $\infty$-topos\footnote{Recall, we may
  characterize an $\infty$-topos as an $\infty$-category $\CC$ such
  that $\CC$ is presentable, colimits are universal, coporducts are
  disjoint, every groupoid object is effective.}. In particular,
$\Shv_{Nis}(\CC)$ is a presentable localization of a presentable
$\infty$-category $\PSh(\CC)$.

The next step is to consider the hypercompletion
$\Shv_{Nis}(\CC)^{\wedge}$ of the $\infty$-topos
$\Shv_{Nis}(\CC)$. Recall the hypercompletion $\XX^{\wedge}$ of an
$\infty$-topos $\XX$ is the left exact localization of $\XX$ at the
$\infty$-connective morphisms, that is, those morphisms which satisfy
a form of Whitehead's theorem in the $\infty$-topos $\XX$ (HTT
pg. 662-663). This implies that $\XX^{\wedge}$ is also an
$\infty$-topos. Equivalently, this is the localization of $\PSh(\CC)$
spanned by objects which are local with respect to the class of
Nisnevich hypercovers.

The final step is to recover $\A^1$-invariance, that is, we want to
restrict to sheaves $F$ such that $F(X) \to F(X \times \A^1)$ is an
equivalence. This is accomplished by taking the localization of
$\Shv_{Nis}(\CC)^{\wedge}$ with respect to the class of all projection
maps $\{X \times \A^1 \to X\}_{X \in \CC}$. Write $\HH(S)$ for the
localization, then from all the general previous nonsense $\HH(S)$ is
presentable with respect to the intermediate universe and very big and
comes with a universal property.

\begin{theorem}{\emph{(\cite{robalo2012noncommutative} Theorem 5.2)}}
  Let $\Sm^{ft}(S)$ be the category of smooth schemes of finite type
  over a Noetherian scheme $S$ and let $L : N(\Sm^{ft}(S)) \to \HH(S)$
  denote the composition of localizations
  \begin{equation*}
    N(\Sm^{ft}(S)) \to \PSh(N(\Sm^{ft}(S))) \to \Shv_{Nis}(\Sm^{ft}(S)) \to \Shv(\Sm^{ft}(S))^{\wedge} \to \HH(S).
  \end{equation*}
  Then for any $\infty$-category $\DD$ with all colimits the map
  induced by composition with $L$
  \begin{equation*}
    \Fun^{L}(\HH(S), \DD) \to \Fun(N(\Sm^{ft}(S)), \DD)
  \end{equation*}
  is fully faithful and has essential image the full subcategory of
  $\Fun(N(\Sm^{ft})(S), \DD)$ spanned by those functors satisfying
  Nisnevich descent (i.e. Theorem~\ref{thm:descent}) and
  $\A^1$-invariance (i.e. $F(X \times \A^1) \to F(X)$ is an
  equivalence) where $Fun^{L}$ is the full subcategory of colimit
  preserving functors.
\end{theorem}

Informally, what the above says is that every colimit preserving
functor $N(\Sm^{ft}(S)) \to \DD$ which satisfies
\begin{itemize}
\item $\A^1$-invariance: $F(X \times \A^1) \to F(X)$ and
\item Nisnevich descent: send elementary distinguished squares to
  pushout squares
\end{itemize}
factors uniquely through $\HH(S)$ which is a presentable
$\infty$-category which we call the unstable motivic
$\infty$-category.

\subsection{The Stable Motivic Homotopy Category}

We now consider how to construct the stable motivic
$\infty$-category. First, $\HH(S)$ is presentable so it has a terminal
object $*$ and $\HH(S)_{*}$ is presentable.

%-------------------------------------------------------------------------------

\section{Algebraic K-theory and Representability in Motivic Homotopy
  Theory}




\addcontentsline{toc}{section}{References}

\nocite{*}
\bibliographystyle{alpha}
\bibliography{motivic.bib}

\printindex

\end{document}
